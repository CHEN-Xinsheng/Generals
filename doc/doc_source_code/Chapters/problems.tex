\section{遇到的困难及解决}

\subsection{初始棋盘信息不正确的问题}
一开始初始棋盘写成 MIF 文件,存在RAM里,用 ip 核读取。但少数情况下读取的少部分数据有问题,猜测可能是 RAM 内存在“噪点”或其它异常。将初始棋盘改成直接用 \texttt{localparam} 写在程序里(打表),可以正常加载初始棋盘。

\subsection{SV 语言特性}
\subsubsection{时序逻辑块中的寄存器赋值问题}

一个寄存器在一个 \texttt{always\_ff} 块中只能赋值一次。如果有多次赋值,那么除最后一个赋值语句之外的语句都将被覆盖。

对于本游戏,在每一回合结束时,既需要结算玩家操作引起的兵力变化,又需要结算兵力的自然增长。为了避免实现代码过于复杂且混乱,我们将这两个部分分离,具体而言即增加 \texttt{ROUND\_SWITCH} 状态,用于结算兵力的自然增长,将两部分兵力变化分开在两个周期内进行。这样同时也让代码的模块化更佳。

\subsubsection{组合逻辑块中的赋值问题}
每个 \texttt{always\_comb}块必须覆盖 \texttt{if} 的所有情况,同时在每种情况下都为变量赋值,否则无法编译。这与平时软件语言的惯例不同,曾导致花费了大量时间来找编译不通过的问题

\subsubsection{运算符语法问题}
Verilog 具有自己的运算语法,尤其是运算符优先级的问题,如移位运算在加法运算之后,这一度导致我们无法正确读取棋盘数据(因为地址运算错误),在查询资料后修改解决。

\subsection{时序问题}
初期,我们选用50MHz时钟用于 $800\times600@75 \, \text{Hz}
$ VGA 显示,100MHz时钟用于游戏内部逻辑运算,发现在早期没有太多逻辑运算时可以维持,但只要增加逻辑就会发生时序报错,画面无法正常渲染。为了避免时序要求不满足的问题,我们采取了两个措施:

\textcircled{1}内部逻辑提速:考虑到复杂性问题,我们将所有的除法(包括取模)和部分乘法用打表实现,这在初期显著提高了性能。

\textcircled{2}更换时钟:在逻辑运算大量增加后,我们意识到原有的时钟频率不能支持这个规模的运算,决定更改时钟,同时修改VGA以适配,选用25MHz时钟用于 $640\times480@60 \, \text{Hz}
$ VGA 显示,50MHz时钟用于游戏内部逻辑运算。调整后,项目时序报错问题得到解决,运行时具有较高的稳定性。