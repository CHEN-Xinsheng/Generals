\section{项目介绍}
\subsection{选题背景}
% 或者叫“选题来源”等

Generals(或译为“将军棋”)
\footnote{游戏链接:\href{https://generals.io/}{https://generals.io/}}
\footnote{开发者页面:\href{https://dev.generals.io/}{https://dev.generals.io/}}
是一款多人对战的实时战略游戏,玩家需要操控兵力,攻城略地,以占领他人王城为最终目标,基本规则简单,但内容不失丰富有趣,更能开发出多种不同的游戏策略。

这款游戏在信息奥赛选手等圈子内具有广泛的知名度,而我们的舍友正是其中的一名达到过世界第一的深度玩家。受他的影响,我们对此游戏产生了浓厚兴趣。考虑到其功能实现的可行性强,我们决定在项目中复现该游戏。




\subsection{游戏规则}

\subsubsection{游戏局面}

游戏在一张 $10\times 10$ 的棋盘上进行,每个格子至少有以下 2 种属性:归属方、格子类型。对于“格子类型”非“山地”的格子,它还有“兵力”属性。

\noindent \textbf{归属方}

归属方即为该格子的拥有者,可以为红方玩家、蓝方玩家或非玩家(下称“NPC”)。
\begin{itemize}
    \item 对于归属方为玩家的格子,其“格子类型”可以为下述的“空地”、“城市”或“王城”。
    \item 对于归属方为NPC的格子,其“格子类型”可以为下述的“空地”、“城市”或“山地”。
\end{itemize}


\noindent \textbf{格子类型}
\begin{itemize}
    \item 空地(普通格子):初始时兵力为 0且归属于NPC。每16回合,所有\textbf{归属于玩家}的空地的兵力+1;归属于NPC的城市的兵力不变。
    \item 城市:初始时兵力为0且属于NPC。每1回合,所有\textbf{归属于玩家}的城市的兵力+1;归属于NPC的城市的兵力不变。
    \item 山地:归属于NPC且没有兵力属性。所有山地格子游戏开始(初始棋局生成)时确定,并自始至终不可被进行任何操作。
    \item 王城:每个玩家有且仅有一个王城,这也是玩家初始占有的唯一格子。每1回合所有王城的兵力+1。当某方玩家的王城被占领(见下方 \ref{subsubsection:operate-rules}“操作规则”)时,该玩家出局,游戏结束。
\end{itemize}

所有格子的格子类型在游戏全过程中不发生改变。

\noindent \textbf{兵力}

“兵力”属性仅“格子类型”非“山地”的格子拥有。该属性是一个非负整数值。在初始时,每方玩家的王城的兵力为9,其它所有格子的兵力均为0。


\subsubsection{操作规则} \label{subsubsection:operate-rules}
双方轮流进行行棋操作。每次操作有一定的时间上限,若达到此上限后仍未行棋,视为放弃本次行棋机会(但在计算行棋次数以及回合数时仍计入)。双方各操作一次(包括超时),计为1回合。

\noindent \textbf{兵力转移机制}

在每次操作中(不妨称操作方为“我方”),可以选择一个我方格子(设该格子的兵力为$n$),将该格子的部分兵力(称为“派出兵力”,其值设为 $d$)转移到一个相邻格子(称为“目标格子”,设其兵力为 $m$)上。转移后,源格子的兵力变为 $n-d$,而目标格子的归属方和兵力依下确定:

\begin{itemize}
    \item 若目标格子归属于我方:目标格子的归属方不变,兵力变为 $m+d$。
    \item 若目标格子归属于对方:发生交战,若 $m < d$,则目标格子的归属方变为我方,兵力变为 $d-m$;否则目标格子的归属方仍为对方,兵力变为 $m-d$。 
    \item 若目标各自归属于 NPC 且非山地:目标格子的归属方变为我方,兵力变为 $d$。
    \item 若目标格子是山地:无法进行此操作。
\end{itemize}

其中,派出兵力 $d$ 可以为 $n-1$ 或 $\lfloor \dfrac{n}{2} \rfloor$,由操作方玩家决定。见第 \ref{subsubsection:operate-modes} 节。

% 【语法】有序列表
% \begin{enumerate}
%   \item 项目 1
%   \item 项目 2
%   \item 项目 3
% \end{enumerate}


\noindent \textbf{操作模式} \label{subsubsection:operate-modes}

玩家使用WASD移动光标,使用SPACE(空格)选中格子(仅可选中己方格子),进入行棋模式。在行棋模式下,玩家可对该格子进行操作,具体而言:按Z可以切换“全移模式”(派出兵力为 $n-1$,默认选择)和“半移模式”(派出兵力为 $\lfloor \dfrac{n}{2} \rfloor$),然后接着使用WASD选择出兵的目标格子。


\noindent \textbf{胜负判断}

当某方玩家的王城被占领时,该玩家出局,游戏结束。

\noindent \textbf{深度限制}

每方玩家有15秒时间进行操作,在剩余5秒时,倒计时数字变红;若超时,则跳过该次操作机会,直接转到对方进行操作。

游戏最多进行999回合,若到达最大回合,对比双方王城兵力,多者获胜;若兵力相同,判为平局。



\subsection{仓库链接}
本项目开发过程通过清华 Git 进行版本控制管理。

清华 Git 仓库链接:\href{https://git.tsinghua.edu.cn/digital-design-lab/2023-spring/digital-design-grp-13}{https://git.tsinghua.edu.cn/digital-design-lab/2023-spring/digital-design-grp-13}。最终版本为 master 分支上的版本。
